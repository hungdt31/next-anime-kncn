\section{Giải pháp}

\subsection{Đối tượng người dùng}
\headindent Đối tượng người dùng chính của website \textit{Next Anime} bao gồm các bạn trẻ và người lớn yêu thích anime, phim hoạt hình và văn hóa Nhật Bản. 
\\
\headindent Đặc biệt, chúng tôi tập trung vào những người dùng từ 15 đến 35 tuổi, những người thường xuyên sử dụng internet để tìm kiếm và xem nội dung giải trí. Ngoài ra, website cũng hướng đến những người mới bắt đầu tìm hiểu về anime, với các hướng dẫn và giới thiệu về các thể loại anime khác nhau.

\subsection{Tính năng nổi bật}
\textit{Next Anime} sẽ cung cấp các tính năng nổi bật như:
\begin{itemize}
\item \textbf{Thư viện phim phong phú:} Cung cấp các bộ phim anime và hoạt hình mới nhất cũng như các tác phẩm kinh điển.
\item \textbf{Giao diện người dùng thân thiện:} Thiết kế đơn giản, dễ sử dụng, cho phép người dùng dễ dàng tìm kiếm và phân loại nội dung.
\item \textbf{Tính năng tương tác:} Cho phép người dùng bình luận, đánh giá và chia sẻ ý kiến về các bộ phim.
\item \textbf{Danh sách yêu thích:} Người dùng có thể lưu lại các bộ phim yêu thích để dễ dàng truy cập sau này.

\end{itemize}
\subsection{Công nghệ nền tảng}
\textit{Next Anime} sẽ được xây dựng trên nền tảng công nghệ hiện đại, bao gồm:
\begin{itemize}
    \item \textbf{Ngôn ngữ lập trình:} Sử dụng JavaScript với framework NextJS để phát triển giao diện người dùng động và tương tác.
    \item \textbf{Backend:} Sử dụng Node.js với Express.js để xây dựng API, xử lý yêu cầu từ người dùng và kết nối với cơ sở dữ liệu.
    \item \textbf{Cơ sở dữ liệu:} Sử dụng MongoDB hoặc PostgreSQL để lưu trữ thông tin về người dùng, phim, và các tương tác của người dùng.
    \item \textbf{Hosting:} Sử dụng dịch vụ đám mây như AWS hoặc Heroku để đảm bảo khả năng mở rộng và tính ổn định của website.
    \item \textbf{Bảo mật:} Áp dụng các biện pháp bảo mật như mã hóa dữ liệu người dùng, bảo vệ API và thực hiện xác thực người dùng để đảm bảo an toàn thông tin.
\end{itemize}
